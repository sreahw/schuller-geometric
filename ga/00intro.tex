Theoretical physics is all about casting our concepts about the real world into rigorous mathematical form, for better or worse. 
But theoretical physical doesn't do that for its own sake.
It does so in order to fully explore the implications of what our concepts about the real world are. So, to a certain extent, the spirit of theoretical physics can be cast into the words of Wittgenstein who said: ``What we cannot speak about [clearly] we must pass over in silence.''
Indeed, if we have concepts about the real world and it is not possible to cast them into rigorous mathematical form, that is usually an indicator that some aspects of these concepts have not been well understood.

Theoretical physics aims at casting these concepts into mathematical language.
But then, mathematics is just that: it is just a language.
If we want to extract physical conclusions from this formulation, we must interpret the language.
That is not the purpose or task of mathematics, that is the task of physicists.
That is where it gets difficult.
But then, again, mathematics is just a language and, going back to Wittgenstein, he said: ``The theorems of mathematics all say the same. Namely, nothing.''
What did he mean by that? Well, obviously, he did not mean that mathematics is useless.
He just referred to the fact that if we have a theorem of the type ``$A$ if, and only if, $B$'', where $A$ and $B$ are propositions, then obviously $B$ says nothing else that $A$ does, and $A$ says nothing else than $B$ does.
It is a tautology. However, while from the point of view of logic and mathematics it is a tautology, psychologically, in terms of our understanding of $A$, it may be very useful to have a reformulation of $A$ in terms of $B$.

Thus, with the understanding that mathematics just gives us a language for what we want to do, the idea of this course is to provide proper language for theoretical physics. In particular, we will provide the proper mathematical language for classical mechanics, electromagnetism, quantum mechanics and statistical physics. We are not going to revise all the mathematics that is needed for these four subjects, but rather we will develop the mathematics from a higher point of view assuming some prior knowledge of these subjects.




























